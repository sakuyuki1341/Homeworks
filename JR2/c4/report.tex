\documentclass{jsarticle}

\usepackage{booktabs}
\usepackage{graphicx}
\usepackage{float}
\usepackage{url}
\usepackage{amsmath}
\usepackage{listings}

\lstset{
  basicstyle={\ttfamily},
  identifierstyle={\small},
  commentstyle={\smallitshape},
  keywordstyle={\small\bfseries},
  ndkeywordstyle={\small},
  stringstyle={\small\ttfamily},
  frame={tb},
  breaklines=true,
  columns=[l]{fullflexible},
  numbers=left,
  xrightmargin=0zw,
  xleftmargin=3zw,
  numberstyle={\scriptsize},
  stepnumber=1,
  numbersep=1zw,
  lineskip=-0.5ex
}

\title{情報領域演習第二 C演習(クラス3)}
\author{学籍番号: 1810678\ クラス: 3 \\
        名前: 山田朔也}

\begin{document}
    \maketitle
    \begin{description}
        \item[課題1]
        \begin{description}
            \item[ソースコード]
            まず、作成したソースコードを、Listing\ref{prob1}に示す。
            \begin{lstlisting}[caption=課題1のソースコード, label=prob1]
                    .data
            newline:.asciiz "\n"
                    .text
                    .globl  main
                    .ent    main
            main:
                    subu    $sp, 16
                    sw      $ra, 12($sp)
            #ここまでおまじない-------------------------------------------------

                    li      $v0, 5
                    syscall                 #read_int
                    move    $t0, $v0

                    li      $t1, 1
                    li      $t2, 1

            k1:     beq     $t0, $t1, k2

                    mult    $t2, $t0
                    mflo    $t2

                    sub     $t0, $t0, 1

                    b       k1

            k2:     move    $a0, $t2
                    li      $v0, 1          #整数を出力する命令
                    syscall                 #$a0の値を画面に表示

                    la      $a0, newline    #$a0にnewlineで定義された文字列をコピー
                    li      $v0, 4          #文字列を出力する命令
                    syscall                 #改行を画面に表示

                    move    $v0, $zero
                    lw      $ra, 12($sp)
                    addu    $sp, 16
                    jr      $ra

                    .end
            \end{lstlisting}

            \item[実行例]
            また、この実行例は以下のListing\ref{prob1_ex}ようになる。
            \begin{lstlisting}[caption=課題1の実行例, label=prob1_ex]
            (spim) load "prob1.asm"
            (spim) run
            4
            24
            (spim) run
            1
            1
            (spim) run
            10
            3628800
            \end{lstlisting}

            \item[考察]
            まず、このコードを実装する際に工夫した点を記述していく。

            工夫した点としてはどれだけ少ない行数で記述できるかという点である。
            ただ、その結果として残念ながら必要であろう機能を一部欠くことになってしまった。
            本当であれば実行例のように"Input a nuber: "や"Output: "などと表示したり、
            LOレジスタの中身だけでなく、値が大きくなったときのためHIレジスタの中身も表示できるように
            するとよりよいプログラムとなることが推測される。
            \\
        \end{description}

        \item[課題2]
        \begin{description}
            \item[ソースコード]
            まず、作成したソースコードを、Listing\ref{prob2}に示す。
            \begin{lstlisting}[caption=課題2のソースコード, label=prob2]
                    .data
            newline:.asciiz "\n"
            hoge:.word  1 4 1 4 2 1 3 5
                    .text
                    .globl  main
                    .ent    main
            main:
                    subu    $sp, 16
                    sw      $ra, 12($sp)
            #ここまでおまじない-------------------------------------------------
                    li      $s0, 0          #ループ用のi
                    li      $s1, 8          #ループ回数の最大数
                    la      $s2, hoge
            k1:     bge     $s0, $s1, k2

                    sll     $s3, $s0, 2     #hoge読み込み準備
                    add     $s3, $s3, $s2
                    lw      $a0, 0($s3)

                    jal     print

                    add     $s0, $s0, 1

                    b       k1

            k2:     move    $v0, $zero
                    lw      $ra, 12($sp)
                    addu    $sp, 16
                    jr      $ra

                    .end
            #ここからサブルーチン-----------------------------------------------
            print:  li      $v0, 1          #整数を出力する命令
                    syscall                 #$a0の値を画面に表示

                    la      $a0, newline    #$a0にnewlineで定義された文字列をコピー
                    li      $v0, 4          #文字列を出力する命令
                    syscall                 #改行を画面に表示

                    jr      $ra
            \end{lstlisting}

            \item[実行例]
            また、この実行例は以下のListing\ref{prob2_ex}ようになる。
            \begin{lstlisting}[caption=課題2の実行例, label=prob2_ex]
            (spim) load "prob2.asm"
            (spim) run
            1
            4
            1
            4
            2
            1
            3
            5
            \end{lstlisting}

            \item[考察]
            まず、今回サブルーチンを使用するため、気をつけなければいけないこととしてmain部にて
            \$t0\~\$t7の使用は控えなければいけない。
            そして、一番重要なのはサブルーチン周りのレジスタの扱いである。ただ、この課題では返り値は
            存在しないため、注意するべきは引数を格納する\$a0レジスタである。
            基本的にこのレジスタに引数を入れることによってサブルーチンに値を渡すため、必ず呼び出し元ではサブルーチンを呼ぶ前に渡したい値を\$a0におさめておかなくてはならない。
            \\
        \end{description}

        \item[課題3]
        \begin{description}
            \item[ソースコード]
            まず、作成したソースコードを、Listing\ref{prob3}に示す。
            \begin{lstlisting}[caption=課題3のソースコード, label=prob3]
                    .data
            newline:.asciiz "\n"
            hoge:.word  1 4 1 4 2 1 3 5
                    .text
                    .globl  main
                    .ent    main
            main:
                    subu    $sp, 16
                    sw      $ra, 12($sp)
            #ここまでおまじない-------------------------------------------------
                    li      $s0, 0          #ループ用のi
                    li      $s1, 8          #ループ回数の最大数
                    la      $s2, hoge
                    li      $s3, 0          #sum
            k1:     bge     $s0, $s1, k2

                    sll     $s4, $s0, 2     #hoge読み込み準備
                    add     $s4, $s4, $s2
                    lw      $a0, 0($s4)

                    jal     func
                    add     $s3, $s3, $v0


                    add     $s0, $s0, 1

                    b       k1

            k2:     move    $a0, $s3
                    jal     print

                    move    $v0, $zero
                    lw      $ra, 12($sp)
                    addu    $sp, 16
                    jr      $ra

                    .end

            #ここからサブルーチン-----------------------------------------------
            func:   move    $v0, $a0

                    mult    $v0, $v0
                    mflo    $v0

                    add     $v0, $v0, 1

                    j       $ra

            print:  li      $v0, 1          #整数を出力する命令
                    syscall                 #$a0の値を画面に表示

                    la      $a0, newline    #$a0にnewlineで定義された文字列をコピー
                    li      $v0, 4          #文字列を出力する命令
                    syscall                 #改行を画面に表示

                    jr      $ra
            \end{lstlisting}

            \item[実行例]
            また、この実行例は以下のListing\ref{prob3_ex}ようになる。
            \begin{lstlisting}[caption=課題3の実行例, label=prob3_ex]
            (spim) load "prob3.asm"
            (spim) run
            81
            \end{lstlisting}

            \item[考察]
            この課題では課題2と異なり、サブルーチンの返り値について考えなくてはならない。
            サブルーチンの返り値はサブルーチン内でまず\$v0レジスタに返したい値を格納し、その後呼び出し元に帰った後\$v0レジスタから取り出して使用することとなる。
            無論\$v0レジスタも汎用レジスタ同様計算に用いることができるため、今回実装したコードでは使用するレジスタを無駄に増やさないよう\$v0を計算に用いている。
            \\
          \end{description}

        \item[課題4]
        \begin{description}
            \item[ソースコード]
            まず、作成したソースコードを、Listing\ref{prob4}に示す。
            \begin{lstlisting}[caption=課題4のソースコード, label=prob4]
                    .data
            newline:.asciiz "\n"
            hoge:   .word  1 4 1 4 2 1 3 5
                    .text
                    .globl  main
                    .ent    main
            main:
                    subu    $sp, 16
                    sw      $ra, 12($sp)
            #ここまでおまじない-------------------------------------------------
                    li      $s0, 0          #ループ用のi
                    li      $s1, 8          #ループ回数の最大数
                    la      $s2, hoge
                    li      $s3, 0          #sum
            k1:     bge     $s0, $s1, k2

                    sll     $s4, $s0, 2     #hoge読み込み準備
                    add     $s4, $s4, $s2
                    lw      $a0, 0($s4)

                    jal     func
                    add     $s3, $s3, $v0


                    add     $s0, $s0, 1

                    b       k1

            k2:     move    $a0, $s3
                    jal     print

                    move    $v0, $zero
                    lw      $ra, 12($sp)
                    addu    $sp, 16
                    jr      $ra

                    .end

            #ここからサブルーチン-----------------------------------------------
            #--関数calc1-------------------------------------------------
            calc1:  li      $t0, 1          #factorial
                    move    $t1, $a0        #i

            k3:     ble     $t1, $zero, k4  #forにおけるi>0
                    mul     $t0,$t0,$t1     #factorial *= i
                    sub     $t1, 1          #i--
                    b       k3

            k4:     move    $v0, $t0        #戻り値を専用レジスタに格納
                    jr      $ra             #親に帰る

            #--関数calc2-------------------------------------------------
            calc2:  li      $t0, 0          #sum
                    li      $t1, 1          #i

            k5:     bgt     $t1,$a0,k6      #forにおけるi<=a
                    add     $t0,$t0,$t1     #sum += i
                    add     $t1, 1          #i++
                    b       k5

            k6:     move    $v0, $t0        #戻り値を専用レジスタに格納
                    jr      $ra             #親に帰る

            #--関数func--------------------------------------------------
            func:   move    $s7, $ra        #戻り番地の保存

                    li      $t0, 2          #条件分岐の準備
                    div     $a0, $t0
                    mfhi    $t0

                    beq     $t0,$zero,toc1  #条件分岐

                    jal     calc2           #calc2呼び出し
                    b       rtmain

            toc1:   jal     calc1           #calc1呼び出し

            rtmain: move    $ra, $s7        #戻り番地を保存先から復元する
                    jr      $ra             #親に帰る

            #--関数print-------------------------------------------------
            print:  li      $v0, 1          #整数を出力する命令
                    syscall                 #$a0の値を画面に表示

                    la      $a0, newline    #$a0にnewlineで定義された文字列をコピー
                    li      $v0, 4          #文字列を出力する命令
                    syscall                 #改行を画面に表示

                    jr      $ra
            \end{lstlisting}

            \item[実行例]
            また、この実行例は以下のListing\ref{prob4_ex}ようになる。
            \begin{lstlisting}[caption=課題4の実行例, label=prob4_ex]
            (spim) load "prob4.asm"
            (spim) run
            74
            \end{lstlisting}

            \item[考察]
            この課題は、課題2,3と異なりサブルーチン内でサブルーチンを呼び出している。その場合困るのは\$raレジスタの値が最後に呼び出されたサブルーチンの戻る先しか保存できていないという点にある。
            ではこれをどのように解決するかというと、今回の課題では一度目に呼び出されたサブルーチンfuncにてサブルーチン呼び出しだ行われても値が保存される\$s0\~\$s7レジスタのどれかに\$a0の内容を保存しておくというものである。
            これをしておく事でサブルーチンからサブルーチンを呼び出したとしても最後に呼び出されたサブルーチンから帰ってきた際に保存しておいた値をもとに戻すことにより呼び出し元に帰ることができる。

            また、仮に呼び出し数が増えたとしても、スタックフレームを利用すれば多くのアドレスを保存しておくことも可能である。
        \end{description}
    \end{description}
\end{document}
