\documentclass{jsarticle}

\usepackage{booktabs}
\usepackage{graphicx}
\usepackage{float}
\usepackage{url}
\usepackage{amsmath}
\usepackage{listings}

\lstset{
  basicstyle={\ttfamily},
  identifierstyle={\small},
  commentstyle={\smallitshape},
  keywordstyle={\small\bfseries},
  ndkeywordstyle={\small},
  stringstyle={\small\ttfamily},
  frame={tb},
  breaklines=true,
  columns=[l]{fullflexible},
  numbers=left,
  xrightmargin=0zw,
  xleftmargin=3zw,
  numberstyle={\scriptsize},
  stepnumber=1,
  numbersep=1zw,
  lineskip=-0.5ex
}

\title{情報領域演習第二 C演習(クラス3) レポート}
\author{学籍番号: 1810678 \\
        名前: 山田朔也}

\begin{document}
  \maketitle
  \noindent
  \large 演習問題
  \begin{description}
    \item[問1]
    まず、作成したソースコードを、Listing\ref{prob1}に示す。
    \begin{lstlisting}[caption=問1のソースコード, label=prob1]
      #include <stdio.h>
      #include <stdlib.h>

      void dump_byte(unsigned int n) {
          int buf[8];
          // 下位から順に2進数にして配列に格納
          for(int i = 0; i < 8; i++) {
              buf[i] = n % 2;
              n = n / 2;
          }
          // 変換結果を上位から順に出力
          for(int i = 7; i >= 0; i--) {
              if(buf[i] == 0) {
                  printf("0");
              } else {
                  printf("1");
              }
          }
      }

      void dump_bytes(unsigned char cp[], int nbyte) {
          for(int i = 0; i<nbyte; i++) {
              dump_byte(cp[nbyte-i-1]);   // little endian -> 逆順
              // dump_byte(cp[i]);    // big endian -> 番地順
              printf(" ");
          }
          printf("\n");
      }

      int main(int ac, char *av[]) {
          float f;    // 単精度浮動小数点数 (32 bit)
          double d;   // 倍精度浮動小数点数 (64 bit)

          // av[1]: プログラム実行の第一引数の文字列
          // atof: 引数で指定した値を倍精度浮動小数点数に変換
          f = atof(av[1]); // 代入の際に単精度に変換される
          d = atof(av[1]); // 倍精度そのまま

          dump_bytes((unsigned char *)&f, sizeof f);
          dump_bytes((unsigned char *)&d, sizeof d);

          exit(0);
      }
    \end{lstlisting}
    また、これの実行結果は以下のListing\ref{prob1_ans}ようになる
    \begin{lstlisting}[caption=問1の実行結果, label=prob1_ans]
    [y1810678@red99 c2]$ ./a.out 1.0
    00111111 10000000 00000000 00000000
    00111111 11110000 00000000 00000000 00000000 00000000 00000000 00000000
    [y1810678@red99 c2]$ ./a.out -1.0
    10111111 10000000 00000000 00000000
    10111111 11110000 00000000 00000000 00000000 00000000 00000000 00000000
    [y1810678@red99 c2]$ ./a.out 6.0
    01000000 11000000 00000000 00000000
    01000000 00011000 00000000 00000000 00000000 00000000 00000000 00000000
    \end{lstlisting}

    \item[問2]
    まず、作成したソースコードを、Listing\ref{prob2}に示す。
    \begin{lstlisting}[caption=問2のソースコード, label=prob2]
    #include <stdio.h>
    #include <stdlib.h>
    #include <stdbool.h>
    int main(int ac, char *av[]) {
    // ここから(-1)^0 * 1.1001 * 2^-4の計算
      int a[] = {1,0,0,1};
      float f = 1;
      for(int i = 0; i < 4; i++) {
          if(a[i] == 1) {
              float temp = 1;
              for(int j = 0; j < i+1; j++) {
                  temp /= 2;
              }
              f += temp;
          }
      }
      for(int i = 0; i < 4; i++) {
          f /= 2;
      }
      printf("%.10f\n", f);

    // --------------------------------------------------------
    // ここから(-1)^0 * 1.10011001 * 2^-4の計算
      int b[] = {1,0,0,1,1,0,0,1};
      double d = 1;
      for(int i = 0; i < 8; i++) {
          if(b[i] == 1) {
              double temp = 1;
              for(int j = 0; j < i+1; j++) {
                  temp /= 2;
              }
              d += temp;
          }
      }
      for(int i = 0; i < 8; i++) {
          d /= 2;
      }
      printf("%.10lf\n", d);
      return 0;
    }
    \end{lstlisting}
    これは、2進数の小数点以下の部分をそれぞれ計算し、足し合わせた後に、必要回数分2で割っているものである。
    また、これの実行結果は以下のListing\ref{prob2_ans}ようになる
    \begin{lstlisting}[caption=問2の実行結果, label=prob2_ans]
    [y1810678@red99 c2]$ ./a.out
    0.0976562500
    0.0062408447
    \end{lstlisting}

    \item[問3]
    まず、作成したソースコードを、Listing\ref{prob3}に示す。
    \begin{lstlisting}[caption=問3のソースコード, label=prob3]
    #include <stdio.h>

    int main(void) {
        double i = 2, j = 3;
        double x = i / j + j / i;
        printf("%.10lf\n", x);
        return 0;
    }
    \end{lstlisting}
    これは、割る数割られる数の両方を実数に変更して計算しているものである。
    また、printfの指定方法を変更して10桁まで表示するようにしている。
    また、これの実行結果は以下のListing\ref{prob3_ans}ようになる
    \begin{lstlisting}[caption=問3の実行結果, label=prob3_ans]
    [y1810678@red99 c2]$ ./a.out
    2.1666666667
    \end{lstlisting}

    \item[問4]
    まず、作成したソースコードを、Listing\ref{prob4}に示す。
    \begin{lstlisting}[caption=問4のソースコード, label=prob4]
    #include <stdio.h>

    int main(void) {
        float x, y;
        x = 777777.7;
        y = 1.111111;
        y *= 10;
        x += y;
        printf("%f\n", x);
        return 0;
    }
    \end{lstlisting}
    これは、10回足すのではなく、1度yに10を掛けてから足している。
    これでもとのコードよりは精度が上がり計算できる。
    また、これの実行結果は以下のListing\ref{prob4_ans}ようになる
    \begin{lstlisting}[caption=問4の実行結果, label=prob4_ans]
    [y1810678@red99 c2]$ ./a.out
    777788.812500
    \end{lstlisting}

    \item[問5]
    まず、作成したソースコードを、Listing\ref{prob5}に示す。
    \begin{lstlisting}[caption=問5のソースコード, label=prob5]
    #include <stdio.h>
    #include <stdlib.h>

    void show_byte(unsigned char x) {
        for(int i = 7; i >= 0; i--) {
            printf("%d", (x>>i) & 1);
        }
        printf("\n");
    }

    int main(void) {
        show_byte(0x6a);
        show_byte(0x0f);
        show_byte((0x6a)|(0x0f));
        show_byte((0x6a)&(0x0f));
        show_byte(~(0x6a));
        return 0;
    }
    \end{lstlisting}
    これは、関数show\_byteに「$(6a)_{16}$」、「$(0f)_{16}$」、「2つの数の論理和」、「論理積」、「$(6a)_{16}$のビット反転させたもの」を引数として渡すことで、それぞれの2進数表現を表示させている。
    また、これの実行結果は以下のListing\ref{prob5_ans}ようになる
    \begin{lstlisting}[caption=問5の実行結果, label=prob5_ans]
    [y1810678@red99 c2]$ ./a.out
    01101010
    00001111
    01101111
    00001010
    10010101
    \end{lstlisting}

    \item[問6]
    まず、作成したソースコードを、Listing\ref{prob6}に示す。
    \begin{lstlisting}[caption=問6のソースコード, label=prob6]
    #include <stdio.h>
    #include <stdlib.h>

    void show_byte(unsigned char x) {
        for(int i = 7; i >= 0; i--) {
            printf("%d", (x>>i) & 1);
        }
        printf("\n");
    }

    int main(void) {
        show_byte((0x11)<<1);
        show_byte((0x11)<<2);
        show_byte((0x11)>>1);
        return 0;
    }
    \end{lstlisting}
    これは、関数show\_byteに$(11)_{16}$をそれぞれ左に1ビットシフトしたもの、左に2ビットシフトしたもの、右に1ビットシフトしたものを引数として渡すことで、それぞれの2進数表現を表示させている。
    また、これの実行結果は以下のListing\ref{prob6_ans}ようになる
    \begin{lstlisting}[caption=問6の実行結果, label=prob6_ans]
    [y1810678@red99 c2]$ ./a.out
    00100010
    01000100
    00001000
    \end{lstlisting}
  \end{description}

  \noindent
  \large 宿題
  \begin{description}
    \item[問題1]
    まず、作成したソースコードを、Listing\ref{prob_hw1}に示す。
    \begin{lstlisting}[caption=問1のソースコード, label=prob_hw1]
    #include <stdio.h>

    int main(void) {
        float a, b, c, d, s;
        a =  100001.0;
        b =  0.123456;
        c =  0.111111;
        d = -100000.0;
        s = 0.0;
        s += a;
        s += d;
        s += b;
        s += c;
        printf("%f\n", s);
    }
    \end{lstlisting}
    これは、先にaとbを計算することで、桁落ちなどを防いでいる。
    また、これの実行結果は以下のListing\ref{prob_hw1_ans}ようになる
    \begin{lstlisting}[caption=問1の実行結果, label=prob_hw1_ans]
    [y1810678@red99 c2]$ ./a.out
    1.234567
    \end{lstlisting}

    \item[問題2]
    まず、作成したソースコードを、Listing\ref{prob_hw2}に示す。
    \begin{lstlisting}[caption=問2のソースコード, label=prob_hw2]
    #include <stdio.h>
    #include <stdlib.h>

    void show_byte(unsigned char x) {
        for(int i = 7; i >= 0; i--) {
            printf("%d", (x>>i) & 1);
        }
        printf("\n");
    }

    int main(void) {
        show_byte((0x6a)&(0xF0));
        return 0;
    }
    \end{lstlisting}
    これは、$(6a)_{16}$と$(1111 0000)_{2}$との論理積をとっているものである。そのため、末尾4ビットが0になる。
    また、これの実行結果は以下のListing\ref{prob_hw2_ans}ようになる
    \begin{lstlisting}[caption=問2の実行結果, label=prob_hw2_ans]
    [y1810678@red99 c2]$ ./a.out
    01100000
    \end{lstlisting}

  \end{description}
\end{document}
