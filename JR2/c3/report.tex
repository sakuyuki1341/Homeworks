\documentclass{jsarticle}

\usepackage{booktabs}
\usepackage{graphicx}
\usepackage{float}
\usepackage{url}
\usepackage{amsmath}
\usepackage{listings}

\lstset{
  basicstyle={\ttfamily},
  identifierstyle={\small},
  commentstyle={\smallitshape},
  keywordstyle={\small\bfseries},
  ndkeywordstyle={\small},
  stringstyle={\small\ttfamily},
  frame={tb},
  breaklines=true,
  columns=[l]{fullflexible},
  numbers=left,
  xrightmargin=0zw,
  xleftmargin=3zw,
  numberstyle={\scriptsize},
  stepnumber=1,
  numbersep=1zw,
  lineskip=-0.5ex
}

\title{情報領域演習第二 C演習(クラス3)}
\author{学籍番号: 1810678\ クラス: 3 \\
        名前: 山田朔也}

\begin{document}
  \maketitle
  \begin{description}
      \item[課題1]
      \begin{description}
          \item[ソースコード]
          まず、作成したソースコードを、Listing\ref{prob1}に示す。
          \begin{lstlisting}[caption=課題1のソースコード, label=prob1]
              .data
              .text
          newline:.asciiz "\n"
              .globl main
              .ent main
          main:
              subu    $sp, 16
              sw      $ra, 12($sp)
              li      $t0, 10
              li      $t1, 15
              li      $t2, 5
              add     $a0, $t0, $t1
              sub     $a0, $a0, $t2
              li      $v0, 1
              syscall
              la      $a0, newline
              li      $v0, 4
              syscall
              move    $v0, $zero
              lw      $ra, 12($sp)
              addu    $sp, 16
              jr      $ra
              .end
          \end{lstlisting}

          \item[実行例]
          また、この実行例は以下のListing\ref{prob1_ex}ようになる。
          \begin{lstlisting}[caption=課題1の実行例, label=prob1_ex]
          (spim) load "prob1.asm"
          (spim) run
          20
          \end{lstlisting}

          \item[解説と考察]
          まずこのコードでは、t0,t1,t2レジスタに計算に必要な各値を入れていく。
          その後add命令とsub命令を用いて求める値を計算してa0レジスタに値を入れる。
          そしてsyscallのサービスの一つである、print\_intを行いa0レジスタ内の整数値を表示、最後に改行を画面に表示してプログラムは終了する。

          また、このコードであるが、まだ使用するレジスタ数を減らす方法がある。
          計算するためにわざわざt0~t2レジスタに値を代入していたが、それ以前からa0レジスタを使用することによって、a0,t0,t1だけで実行したい計算を行うことができるだろう。
          また、即値演算命令を用いればa0レジスタのみで計算を行うことも可能である。
          \\
      \end{description}

      \item[課題2]
      \begin{description}
          \item[ソースコード]
          まず、作成したソースコードを、Listing\ref{prob2}に示す。
          \begin{lstlisting}[caption=課題2のソースコード, label=prob2]
              .data
          newline:.asciiz "\n"        #改行を画面に表示するおまじない
              .text
              .globl main
              .ent main
          main:
              subu    $sp, 16
              sw      $ra, 12($sp)
              li      $t0, 9
              li      $t1, 2
              div     $t0, $t1

              mflo    $a0
              li      $v0, 1          #整数を出力する命令
              syscall                 #$a0の値を画面に表示

              la      $a0, newline    #$a0にnewlineで定義された文字列をコピー
              li      $v0, 4          #文字列を出力する命令
              syscall                 #改行を画面に表示

              mfhi    $a0
              li      $v0, 1
              syscall

              la      $a0, newline
              li      $v0, 4
              syscall

              move    $v0, $zero
              lw      $ra, 12($sp)
              addu    $sp, 16
              jr      $ra
              .end
          \end{lstlisting}

          \item[実行例]
          また、この実行例は以下のListing\ref{prob2_ex}ようになる。
          \begin{lstlisting}[caption=課題2の実行例, label=prob2_ex]
          (spim) load "prob2.asm"
          (spim) run
          4
          1
          \end{lstlisting}

          \item[解説と考察]
          まずこのコードでは、t0,t1レジスタに計算に必要な各値を入れていく。
          その後div命令用いて求める値を計算して,LOレジスタとHIレジスタの中身をa0レジスタに入れ、表示していく。

          今回の課題は前課題と異なり、使用するレジスタをこれ以上減らすことはできないだろう。
          なぜならdiv命令は即値による演算ができず、値はレジスタを経由して指定するしかないからである。
          また、move命令は汎用レジスタ名しか指定できないため、HI,LOからデータを移す時はmflo,mfhiを利用することに注意が必要だと思われる。
          \\
      \end{description}

      \item[課題3]
      \begin{description}
          \item[ソースコード]
          まず、作成したソースコードを、Listing\ref{prob3}に示す。
          \begin{lstlisting}[caption=課題3のソースコード, label=prob3]
              .data
          newline:.asciiz "\n"        #改行を画面に表示するおまじない
              .text
              .globl main
              .ent main
          main:
              subu    $sp, 16
              sw      $ra, 12($sp)

              li      $v0, 5
              syscall                 #read_int
              move    $t0, $v0

              li      $v0, 5
              syscall                 #read_int
              move    $t1, $v0

              add     $a0, $t0, $t1

              li      $v0, 1          #整数を出力する命令
              syscall                 #$a0の値を画面に表示

              la      $a0, newline    #$a0にnewlineで定義された文字列をコピー
              li      $v0, 4          #文字列を出力する命令
              syscall                 #改行を画面に表示

              move    $v0, $zero
              lw      $ra, 12($sp)
              addu    $sp, 16
              jr      $ra
              .end
          \end{lstlisting}

          \item[実行例]
          また、この実行例は以下のListing\ref{prob3_ex}ようになる。
          \begin{lstlisting}[caption=課題3の実行例, label=prob3_ex]
          (spim) load "prob3.asm"
          (spim) run
          150
          12
          162
          \end{lstlisting}

          \item[解説と考察]
          まずこのコードでは、syscallのサービスの一つのread\_intを用いて2つの数値をt0,t1レジスタにそれぞれ代入していく。
          その後add命令用いて求める値を計算して,その結果をa0レジスタに入れ、表示していく。

          今回の課題は前課題と異なり、逆にmove命令を利用するのが良いだろう。
          また、その他にもadd命令を利用して代入する方法もあるが、moveが一番単純明快だと思われる。
          \\
      \end{description}

      \item[課題4]
      \begin{description}
          \item[ソースコード]
          まず、作成したソースコードを、Listing\ref{prob4}に示す。
          \begin{lstlisting}[caption=課題4のソースコード, label=prob4]
              .data
          newline:.asciiz "\n"        #改行を画面に表示するおまじない
          hoge:.word  1 4 1 4 2 1 3 5
          bar:.word 0 0 0 0 0 0 0 0
              .text
              .align 2
              .globl main
              .ent main
          main:
              subu    $sp, 16
              sw      $ra, 12($sp)

              la      $t0, hoge
              la      $t1, bar
              li      $t2, 0          #i
              li      $t3, 8          #定数
              li      $t4, 0          #sum

          k1: bge     $t2, $t3, k2

              sll     $t5, $t2, 2
              add     $t6, $t5, $t1
              add     $t5, $t5, $t0
              lw      $t7, 0($t5)

              add     $t4, $t4, $t7

              sw      $t4, 0($t6)

              lw      $t7, 0($t7)

              move    $a0, $t7

              li      $v0, 1          #整数を出力する命令
              syscall                 #$a0の値を画面に表示

              la      $a0, newline    #$a0にnewlineで定義された文字列をコピー
              li      $v0, 4          #文字列を出力する命令
              syscall                 #改行を画面に表示

              add     $t2, $t2, 1
              b       k1

          k2: move    $v0, $zero
              lw      $ra, 12($sp)
              addu    $sp, 16
              jr      $ra

              .end
          \end{lstlisting}

          \item[実行例]
          また、この実行例は以下のListing\ref{prob4_ex}ようになる。
          \begin{lstlisting}[caption=課題4の実行例, label=prob4_ex]
          (spim) load "prob4.asm"
          (spim) run
          1
          5
          6
          10
          12
          13
          16
          21
          \end{lstlisting}

          \item[解説と考察]
          まずこのコードでは、hogeとbarという名前の配列を作成し、変数i,sumと定数用のレジスタを確保する。
          次にループの中身を実装で、まず最初に条件分岐命令を用いてループを実行するかどうかを判定する。
          その後、sllを用いて、配列が0番目からi番目までのアドレスのズレを計算してt5レジスタに保存しておく。
          また、これに乗じてbarのi番目のアドレスをt6レジスタに、hogeのi番目のアドレスをt5レジスタに保存する。
          そしてsumの値を計算した後barのi番目に書き込み、これをもう一度t7にロードし表示して正しく書き込めているかを確かめる。

          この課題であるが、一番重要なのはきちんと書き込めているかを、再度ロードして確かめる部分である。
          ただ、読み込み書き込みのテストのための課題なため今回はこのコードが正しいが、本来であれば余計な処理であるため
          再読込みをせずにそのままsumを書き出すのが処理速度を考える上で重要となるだろう。
          
      \end{description}
  \end{description}
\end{document}
