\documentclass{jsarticle}

\usepackage{booktabs}
\usepackage{graphicx}
\usepackage{float}
\usepackage{url}
\usepackage{amsmath}
\usepackage{listings}

\lstset{
  basicstyle={\ttfamily},
  identifierstyle={\small},
  commentstyle={\smallitshape},
  keywordstyle={\small\bfseries},
  ndkeywordstyle={\small},
  stringstyle={\small\ttfamily},
  frame={tb},
  breaklines=true,
  columns=[l]{fullflexible},
  numbers=left,
  xrightmargin=0zw,
  xleftmargin=3zw,
  numberstyle={\scriptsize},
  stepnumber=1,
  numbersep=1zw,
  lineskip=-0.5ex
}

\title{情報領域演習第二 L演習(クラス3) レポート}
\author{学籍番号: 1810678 \\
        名前: 山田朔也}

\begin{document}
  \maketitle
  \begin{description}
      \item[問一]
      \begin{description}
          \item[(a)]
          まず、与えられた論理式$f$の否定$\overline{f}$を計算し、それを積和標準形に変形する。その後、積和標準形で表された論理式$\overline{f}$のさらに否定$\overline{\overline{f}}$を計算することで、和積標準形に変換することができる。これらの計算は全てド・モルガンの法則を適用し、分配律に沿って計算することで求めることが可能である。

          \item[(b)]
          \begin{description}
              \item[i.]
              まず、与えられた論理式$f_1$の否定$\overline{f_1}$を計算する
              \begin{align}
                  \overline{f_1} &= \overline{(x\overline{y}\overline{z} + \overline{x}y\overline{z} +  \overline{x}\overline{y}z)} \notag \\
                                 &= \overline{(x\overline{y}\overline{z})} \cdot \overline{(\overline{x}y\overline{z})} \cdot \overline{(\overline{x}\overline{y}z)} \notag \\
                                 &= (\overline{x}+y+z) \cdot (x+\overline{y}+z) \cdot (x+y+\overline{z}) \notag \\
                                 &= \overline{x}\overline{y}\overline{z} + \overline{x}yz + xyz + xy\overline{z} + x\overline{y}z
              \end{align}
              この
          \end{description}
      \end{description}

  \end{description}
\end{document}
